\section{function pointers}


\subsection{Grundlagen, bind}

\begin{frame}{function pointers}
	\footnotesize
	
	\begin{block}{Allgemeines}
		\begin{itemize}
			\item ein \enquote{normaler} Pointer enthält die Adresse eines »Dinges«
			\item ein \emph{function pointer} enthälten die Adresse einer Funktion
			\item Nicht alle Funktionen haben eine Adresse! (z.B. ctor)
			\item Signatur der Funktion gibt Typen des function ptrs
		\end{itemize}
	\end{block}
	
	\pause
	
	Syntax grausam! Nicht direkt verwenden!
	
	\begin{block}{function objects}
		\begin{itemize}
			\item function objects können der (syntaktisch) wie eine Funktion aufgerufen werden (operator overloading)
			\item enthalten entweder selbst den Funktionsaufruf oder sind wrapper für function ptrs
			\item für optimale Performance kein spezifizierter Typ / template
			\item Erzeugung und Speicherung am besten mittels Automatisierungen
		\end{itemize}
	\end{block}
\end{frame}

\begin{frame}[fragile]{Beispiel: function objects}
	\begin{lstlisting}
		int doCalc1(int, double, bool);
		int doCalc2(int, double, bool);
		int doCalc3();
		
		
		#include <functional>
		
		auto myFunc = std::bind( &doCalc1 );	// OK
		     myFunc = std::bind( &doCalc2 );	// OK
		     myFunc = std::bind( myFunc );		// OK
		     myFunc = std::bind( &doCalc3 );	// error!
		
		int  result = myFunc(4, 2.0, true);
	\end{lstlisting}
\end{frame}

\begin{frame}[fragile]{ \texttt{std::bind} \& ownership }
	\begin{itemize}
		\item \texttt{bind} kopiert üblicherweise
		\item moved, wenn kopieren verboten
		\item alternativ: \texttt{std::ref} oder (smart) ptr
	\end{itemize}
	
	\pause
	
	\begin{lstlisting}
		void thieve( unique_ptr < MonaLisa > );
		bool drawAt( MonaLisa& );
		
		MonaLisa ml;
		std::bind( &drawAt, std::ref(ml) );
		
		unique_ptr < MonaLisa > pML = new MonaLisa;
		std::bind( &thieve, pML );
	\end{lstlisting}
\end{frame}

\begin{frame}[fragile]{ placeholders }
	\begin{lstlisting}
		int doCalc1(int, bool, double);
		
		#include <functional>
		
		auto myFunc = std::bind( &doCalc1, 42, _1, _2 );
		//- myFunc has signature `int(bool, double)`
		
		int  result = myFunc(true, 4.2);
	\end{lstlisting}
	
	\pause
	
	\begin{lstlisting}[firstnumber=9]
		auto myFunc2 = std::bind( &doCalc1, _1, false, 21.42);
		//- myFunc has signature `int (int)`
		result = myFunc2(0);
	\end{lstlisting}
\end{frame}


\subsection{ member functions }

\begin{frame}[fragile]{ member functions }
	\begin{itemize}
		\item Bisher: »Ding« pointer und function pointer
		\item Es gibt aber auch pointer-to-member (extrem selten!) und pointer-to-member-function \dots
		\item Veranschaulichung: Auf welcher Instanz soll die member function aufgerufen werden, wenn ich nur ihre Adresse habe?
		\item Veranschaulichung: Wer setzt den \texttt{this}-Pointer?
	\end{itemize}
	
	\pause
	
	Normale Syntax noch hässlicher als die von function pointers!	\\
	Daher wieder mit function objects.
\end{frame}

\begin{frame}[fragile]{ member functions: Beispiel }
	\begin{lstlisting}[basicstyle=\scriptsize]
		struct Calculator {
		    int add(int);
		    int sub(int);
		    int res() const;
		};
		struct Calc2 {
		    int add(int);
		};
		
		auto myFunc = std::bind( &Calculator::add );
		//- myFunc has signature akin to `int (Calculator&, int)`
		     myFunc = std::bind( &Calculator::sub );
		     myFunc = std::bind( &Calculator::res );    // error!
		     myFunc = std::bind( &Calc2::add );         // error!
		
		shared_ptr < Calculator > spC = new Calculator;
		Calculator& rC = *spC;
		Calculator* pC = spC.get();
		
		int result = myFunc(spC, 5);
		    result = myFunc(rC, 5);
		    result = myFunc(pC, 5);
	\end{lstlisting}
\end{frame}

\begin{frame}[fragile]{ member function to (free) functions }
	Man kann die zu verwendene Instanz an das function object binden!
	
	\begin{lstlisting}
		Calculator myCalc;
		auto myFunc = std::bind( &Calculator::add, myCalc );
		//- myFunc has signature `int(int)`
		     myFunc = std::bind( &Calculator::add, std::ref(myCalc) );
		     myFunc = std::bind( &Calculator::add, &myCalc );
			 
		shared_ptr < Calculator > pMyCalc = new Calculator;
		     myFunc = std::bind( &Calculator::add, pMyCalc );
			 
		int result = myFunc(42);
	\end{lstlisting}
\end{frame}


\subsection{ polymorphic function wrappers }

\begin{frame}[fragile]{ polymorphic function wrappers }
	\footnotesize
	
	Wie bei pointern + Polymorphismus kann in einem function object alles mögliche referenziert werden (z.B. member function vs. normale function).	\\
	$\implies$ spezielles wrapper-Objekt, das eben alles mögliche Aufnehmen \& speichern kann (auch gebundene Parameter usw.)
	
	\pause
	
	\begin{lstlisting}[basicstyle=\scriptsize]
		#include <functional>
		
		struct EventProvider {
		    std::function < int(bool) > observer;
		    void pollEvent() {
		        /* ... */
		        int result = observer(true);
		        /* ... */
		    }
		};
		
		int onEvent(bool);
		
		EventProvider ep;
		ep.observer = std::bind( &onEvent );
		/* ... */
		ep.pollEvent();
	\end{lstlisting}
\end{frame}
