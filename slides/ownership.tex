\section{ownership of data}


\subsection{C++ and copying}

\begin{frame}[fragile]{assignment}
	\begin{block}{Philosophie}
		\begin{itemize}
			\item Standard, 5.17/2: Zuweisung ersetzt Wert der linken Seite durch Wert der rechten Seite
			\item Standard, 12.8/29: Die default-Zuweisung ist eine member-weise Zuweisung (zunächst Zuweisung der base-class subobjects, dann der der member)
		\end{itemize}
		
		$\implies$ Kopie $\implies$ \enquote{copy-assignment}
	\end{block}
	
	\pause
	
	\begin{lstlisting}
		struct MyS {
		    public:
		    /*inline*/ MyS& operator= (MyS const&);
		};
		
		MyS a;
		MyS b;
		a = b;
	\end{lstlisting}
\end{frame}

\begin{frame}[fragile]{copy-ctor}
	\begin{block}{Philosophie}
		\begin{itemize}
			\item Standard, 12.1/9: der copy-ctor kopiert »Dinge«
			\item Standard, 12.8/16: Der default-copy-ctor kopiert member-weise (zunächst Kopieren der base class subobjects)
		\end{itemize}
		
		$\implies$ Kopie
	\end{block}
	
	\begin{lstlisting}
		struct MyS {
		    public:
		    /*inline*/ MyS(MyS const&);
		};
		
		MyS a;
		MyS b(a);
		// oder
		MyS b = a;
	\end{lstlisting}
\end{frame}

\begin{frame}[fragile,allowframebreaks]{Grenzen des Kopierens}
	\begin{itemize}
		\item Referenzen als data members
		\item Pointer + RAII
		\item const data members
		\item class-type data member ohne assignment-op/copy-ctor
		\item exception-safety
	\end{itemize}
	
	In diesen Fällen (bis auf ptr) wird auch kein default copy-ctor/assignment-op definiert! (dies schlägt fehl)
		
	\begin{block}{Achtung!}
		Der Compiler darf das Kopieren von temporaries vermeiden, selbst wenn assignment-op/copy-ctor Seiteneffekte haben!\\
		Beispiele: function arguments, return value
	
		\begin{lstlisting}
			MyS foo(MyS p) {
			    MyS ret( p );
			    return ret;
			}
		
			MyS a;
			MyS b = foo(a);
		\end{lstlisting}
	\end{block}
\end{frame}

\begin{frame}[fragile]{swap}
	\begin{block}{ assignment mit \emph{strong guarantee} }
		\begin{lstlisting}
			struct My {
			    My& operator= (My const& rhs) {
			        Me_can_throw temp;
			        temp = rhs.member;	// copy, can and may throw
					
			        std::swap(temp, this->member);	// swap, no-throw!
			        return *this;
			    }
			private:
			    Me_can_thow member;
			};
		\end{lstlisting}
	\end{block}
	
	\pause
	
	\begin{block}{pro/contra von swap gegenüber doppelter Kopie}
		\begin{itemize}
			\item[+] meist no-throw (\texttt{noexcept})
			\item[+] bei STL-Containern in $O(1)$
			\item[-] expliziter Aufruf notwendig
			\item[-] kein swap-ctor möglich
		\end{itemize}
	\end{block}
\end{frame}

\begin{frame}[fragile]{move semantics}
	\begin{block}{Philosohpie}
		\begin{itemize}
			\item Philosophie: (12.1/9) Verschieben von Inhalten
			\item 12.8/16, 12.8/29: default move-assignment \& move-ctor verschieben member-weise
		\end{itemize}
	\end{block}
	
	\pause
	
	\begin{lstlisting}
		struct MyS {
		    public:
		    /*inline*/ MyS(MyS&&);
		    /*inline*/ MyS& operator= (MyS&&);
		};
		
		MyS b = MyS();
		b = MyS();
		
		MyS a;
		b = std::move(a);
	\end{lstlisting}
\end{frame}

\begin{frame}[fragile]{Beispiel}
	\begin{lstlisting}
		vector < int > a = {0, 1, 2, 3};
		vector < int > b = {0, 1, 2, 3, 4};
		vector < int > b_pre;
		
		b_pre = b;
		//- b_pre == {0, 1, 2, 3, 4}
		a = move(b);
		//- a == b_pre
		//- b == >undefined<
	\end{lstlisting}
	
	\pause
	
	Kopieren: Elementweises Kopieren ($\implies O(n)$) \\
	Verschieben: Pointer tauschen ($\implies O(1)$)
\end{frame}

\begin{frame}{STL data structures}
	\begin{itemize}
		\item haben ein part-whole-ownership wie Elemente\textless-\textgreater Menge
		\item hauptsächlich: default-ctor, kopieren \\
		      bspw.: \texttt{resize + []} vs. \texttt{reserve + push\_back + lvalue ref}
		\item Vermeiden von Kopien: move semantics oder Pointer \\
		      bspw: \texttt{emplace\_back}
		\item Problem bei raw ptrs: bei Zerstörung des Pointers wird das Ziel nicht zerstört!
		\begin{itemize}
			\item Daher: Ungeeignet für das ownership-Modell der STL-Container\\
			      (aber möglich)
		\end{itemize}		      
	\end{itemize}
\end{frame}


\subsection{ownership \& smart pointer}

\begin{frame}[fragile]{Zwei Probleme (1)}
	\begin{block}{Problem 1}
		\begin{lstlisting}
			int* p = new int[42];
			
			if( cond ) {
			    delete[] p;
			    return;
			}
			
			try{ something() }
			catch(...) {
			    delete[] p;
			    throw;
			}
			
			try{ something() }
			catch(MyExpT&)  { delete[] p; }
			catch(MyExpT1&) { delete[] p; }
			catch(...)      { delete[] p; }
		\end{lstlisting}
	\end{block}
\end{frame}

\begin{frame}[fragile]{Zwei Probleme (2)}
	\begin{block}{Problem 2}
		\begin{lstlisting}[basicstyle=\scriptsize]
			struct Louvre {
			    MonaLisa* show();
			    ~Louvre() {
			        delete pML;
			    }
			private:
			    MonaLisa* pML;
			};
			
			struct Peruggia {
			    void thieve(MonaLisa* p) {
			        pML = p;
			    }
			    ~Peruggia() {
			        delete pML;
			    }
			private:
			    MonaLisa* pML;
			};
		\end{lstlisting}
	\end{block}
\end{frame}

\begin{frame}{ smart pointers }
	\begin{itemize}
		\item wrapper-Objekte für (raw) Pointer
		\item RAII-Konstrukte (beim Zerstören wird automatisch aufgeräumt)
		\item explizites owenership-Modell
		\item exception-\enquote{safe}
		\item alter Repräsentant: auto\_ptr -- nicht verwenden (deprecated)!
		\item Verwendung mit synax analog Pointern (\texttt{*} dereferenzieren, \texttt{->} (operator overloading)
	\end{itemize}
\end{frame}

\begin{frame}[fragile]{ \texttt{unique\_ptr} }
	\begin{itemize}
		\item \textbf{ein} Besitzer
		\item move-semantics zum Besitzerwechsel
		\item thread-safe movement
		\item einzelne Instanz \emph{nicht} thread-safe
	\end{itemize}
	
	\begin{lstlisting}
		#include <memory>
		
		unique_ptr < int > myInt = new int;
		unique_ptr < int > otherInt = new int;
		
		otherInt = myInt;	// copying is forbidden! (makes no sense)
		
		*otherInt = 5;
		cout << *otherInt;
		
		vector < unique_ptr < LargeData > > myVec;	// ok
	\end{lstlisting}
\end{frame}

\begin{frame}[fragile]{ \texttt{unique\_ptr} \& move }
	Mit move-semantic kann das Besitzer-Objekt geändert werden:
	
	\begin{lstlisting}[basicstyle=\tiny]
		struct Louvre {
		    MonaLisa& show();
		    unique_ptr < MonaLisa > breakIn();
			
		    ~Louvre() {}
		private:
		    unique_ptr < MonaLisa > pML;
		};
		
		struct Peruggia {
		    void thieve(unique_ptr < MonaLisa > p) {
		        pML = move(p);
		    }
		    ~Peruggia() {}
		private:
		    unique_ptr < MonaLisa > pML;
		};
		
		Louvre myMuseum;
		Peruggia myThief;
		
		unique_ptr < MonaLisa > pML = move( myMusee.breakIn() );
		myThief.steel( move(pML) );
	\end{lstlisting}
\end{frame}

\begin{frame}[fragile]{ \texttt{shared\_ptr} }
	\begin{itemize}
		\item viele, gleichrangige Besitzer (wie Taxi-Gemeinschaft)
		\item der letzte macht das Licht aus (ruft \texttt{delete} auf)
		\item thread-safe Besitzer-Operationen (z.B. Erzeugen von Mitbesitzer-Instanzen)
		\item einzelne Instanz \emph{nicht} thread-safe
	\end{itemize}
	
	\begin{lstlisting}[basicstyle=\scriptsize]
		#include <memory>
		
		shared_ptr < int > myInt = new int;
		{
		    shared_ptr < int > otherInt = new int;
			
		    otherInt = myInt;	// ok, old int of otherInt is destroyed
			
		    *otherInt = 5;
		    cout << *otherInt;
		}//- otherInt is destroyed, but its content lives on
		
		vector < shared_ptr < LargeData > > myVec;	// ok
	\end{lstlisting}
\end{frame}

\begin{frame}[fragile]{ \texttt{shared\_ptr} \& copying }
	Mit dem shared\_ptr lässt sich oft Kopieren vermeiden:\\
	Alle, die Zugriff auf die Daten brauchen, werden einfach Mitbesitzer (indem sie eine shared\_ptr-Instanz speichern).
	
	\begin{lstlisting}
		struct A { shared_ptr < foo > m; };
		struct B { shared_ptr < foo > m; };
		
		void f(shared_ptr < foo > p) {
		    /* do something */
		}
		
		shared_ptr < foo > p = new foo;
		A a = {p};
		B b = {p};
		f(p);
		
		b.m.reset();
		p.reset();
		a.m.reset();
	\end{lstlisting}
\end{frame}

\begin{frame}[fragile]{Der kleine Kommunismus in C++}
	Niemals raw pointer und smart\_ptr mischen!
	
	\begin{lstlisting}
		shared_ptr < foo > sp = p;
		delete p;
	\end{lstlisting}
	
	\pause
	
	Defensives Programmieren mit shared\_ptr: Factory
	
	\begin{lstlisting}
		class My {
		private:
		    My();
			
		public:
		    typedef shared_ptr < My > Ptr;
			
		    static PMy create() {
		        return new My;
		    }
		};
		
		My* p = new My; // forbidden
		My::Ptr p = My::create(); // OK
	\end{lstlisting}
\end{frame}

\begin{frame}{ \texttt{shared\_ptr} \& Hierarchie }
	Situation:
	\begin{itemize}
		\item Ein Besitzer-Objekt \texttt{}r hält ein Besitztum \texttt{d} in beliebigem ownership.
		\item \texttt{r} selbst wird in shared ownership gehalten.
		\item \texttt{d} benötigt einen Zugang zu \texttt{r}.
	\end{itemize}
	
	\pause
	Möglichkeit 1: \texttt{d} hält einen shared\_ptr auf \texttt{r}\\
	\pause
	Möglichkeit 2: \texttt{d} hält einen raw ptr auf \texttt{}r\\
	\pause
	Lösung: weak\_ptr
	
	\begin{block}{weak\_ptr}
		\begin{itemize}
			\item hat selbst keinen Besitz am ptr
			\item kein Dereferenzieren usw.
			\item kann in einen shared\_ptr verwandelt werden
			\item kann auf Zerstörtheit geprüft werden
		\end{itemize}
	\end{block}
\end{frame}

\begin{frame}[fragile]{ \texttt{weak\_ptr}: Beispiel }
	\begin{lstlisting}
		struct Owned;
		struct Owner {
		    shared_ptr < Owned > m;
		    ~Owner();
		};
		
		struct Owned {
		    weak_ptr < Owner > m;
		};
		
		Owner::~Owner() {}
	\end{lstlisting}
\end{frame}
